\documentclass{ExpressiveResume}
\usepackage{hyperref}
\usepackage{fontawesome5}
\usepackage{tabularx}
\usepackage{enumitem}
\usepackage{etoolbox}
\patchcmd{\thebibliography}{\section*{\refname}}{}{}{}


% Increase the space between citation number and text

\setlist{nolistsep}

% ----- Resume -----
\begin{document}


% ----- Name + Contact Information -----
\resumeheader[
    firstname=Alberto,
    lastname=Rota,
    email=alberto\_rota@outlook.com,
    linkedin=albe-rota,
    github=alberto-rota,
    city=Milan,
    state=Italy,
]

% ----- Education -----
\section{Education}

\experience{Ph.D in Bioengineering}{Politecnico di Milano \& Asensus Surgical Inc., Milan, IT}{Feb 2023}{\textit{Ongoing}}{
\noindent Focus: Computer Vision applications for enhanced spatial context awareness in surgical robotics
}
\nocite{ziopporco}

\vspace{0.1cm}

\experience{MSc in Bioengineering}{Politecnico di Milano, Milan, IT}{Sep 2020}{Dec 2022}{
\noindent Focus: AI and Computer Vision methods for 3D data in bioengineering; Virtualization of teleoperated surgical robotic environments
}

\vspace{0.1cm}

\experience{Visiting Student}{Université de Liège, Liège, BE}{Feb 2022}{Jun 2022}{
Focus: Finite Element Analysis, Robotics
}
% ----- Work Experience -----
\section{Work Experience}

\experience{Ph.D Student researcher}{Asensus Surgical Inc.}{Feb 2023}{\textit{Ongoing}}{
Focus: Computer Vision Deep Learning methods for enhancing the spatial and contextual informative content of endoscopic image data, with focus on 3D reconstruction and occlusion restoration
\achievement{
    Learned, applied and deployed state-of-the-art models, frameworks and pipeline targeted at recovering 3D information from 2D endoscopic image data, with strong focus on self-supervised feameworks [NDA]
}
\achievement{
    Worked in structured teams, both in a contributing and leading position. Mastered project management, time management and DevOps skills
}
}
\experience{Teaching Assistant}{NEARLab MRS}{Sep 2023}{\textit{Ongoing}}{
    Subject: Technologies for Motor Behavior Analysis and Virtual Modelling
\achievement{Mastered communication and public speaking skills}
}
\vspace{-0.1in}
% ----- Tech Stack -----
\section{Tech Stack}
\noindent\begin{tabular}{@{} >{\raggedleft}p{0.1\linewidth} p{0.3\linewidth} >{\raggedleft}p{0.1\linewidth} p{0.375\linewidth}}
\textit{Coding}\phantom{..} & Python, C++, C, C\#, MATLAB & \textit{CAD}& Autodesk Inventor, Blender\\
\textit{ML}\phantom{..} & PyTorch, Lighting, SciKit & \textit{WebDev} & HTML5, Wordpress \href{https://nearlab.polimi.it/medical/}{\faLink[regular]} \\
\textit{DevOps}\phantom{..} & Git, Docker, Slurm & \textit{Office} & \LaTeX, MS Office Suite\\
\textit{MLOps}\phantom{..} & WandB   & \textit{Graphics}& Figma, Inkscape \\
\textit{Misc}\phantom{..} & ROS, OpenFOAM, Unity  & & \\
\end{tabular}

% ----- Technical Projects -----
\vspace{0.1cm}
\section{Technical Projects}

\experience{µVES}{Academic Research}{Mar 2020}{Jul 2022}{
A fully automated algorithm for the topo-morphological analysis of 3D microvascular networks images from confocal microscopy, with DL-based confocal image segmentation \cite{rota2023three} \href{https://github.com/alberto-rota/muVES}{\faGithub}
\achievement{
    Built and trained a 3D U-Net for segmentation of 3D images from confocal microscopy. 
}
\achievement{
    Developed a complete pipeline for quantitative analysis inclusive of segmentation, skeletonization, and morphological measurement
}
\achievement{
    Primarily contributed and lead a team of 4 researchers, mastering problem-solving and leadership skills 
}
}
\experience{STEVE - Surgical Training Enhanced Virtual Environment}{Master Thesis}{Feb 2022}{Dec 2020}{
A virtual training environment targeting teleoperated surgical robotics for learning key surgical skills, enhanced with visuo-haptic assistance-as-needed guidance, personalized adaptive difficulty and visual feedback for haptic force training \href{https://github.com/alberto-rota/STEVE}{\faGithub}
\achievement{
    Built and validated a VR simulator for surgical robotics in Unity, connected via ROS to a teleoperation console. Developed haptic assistance-as-needed guidance algorithms \cite{rota2023implementation} 
}
\achievement{
    Supervised MSc students on the development and integration of surgical tasks with morpho-adaptive difficulty \cite{rota2023adaptive} and visual feedback for grasping force training
}
}
\experience{Ground Control}{Open-source Python Package}{Dec 2024}{\textit{Maintain.}}{
A Terminal-based package for monitoring system hardware in real time with rich plots and graphics in the terminal. Aimed for multi-GPU machines and ML development.  
\href{https://github.com/alberto-rota/ground-control}{\faGithub} \& \href{https://pypi.org/project/ground-control-tui/}{\texttt{PyPI}}
}
\experience{CT Image SuperResolution}{Academic didactic project}{Oct 2021}{Jan 2022}{
A Deep Learning Model for denoising and super-resoution of CT scans. The model is a convolutional residual architecture trained with a self-supervised routine \href{https://github.com/alberto-rota/CT-SuperResolution-with-Deep-Learning}{\faGithub}

}

\experience{ECG Heartbeat LSTM Classifier}{Academic didactic project}{Oct 2021}{Jan 2022}{
A data-driven classifier for Normal, Sopraventricular and Ventricular heartbeats from ECG signals based on LSTMs, reaching an F1 score of 96\% on the test set \href{https://github.com/alberto-rota/PAC-PVC-Beat-Classifier-for-ECGs}{\faGithub}
\achievement{
    Learned and applied Deep Learning architectures for multivariate time series data
}
\achievement{
    Applied team-working skills
}
}

\vspace{0.1cm}
% ----- Awards -----
\section{Awards}
\experience{Best Application Award}{Hamlyn Surgical Robotics challenge 2023}{}{Jun 2023}{
Haptic assistance for improving skill transfer in surgical robotics training \href{https://www.hamlynsymposium.org/surgical-robot-challenge-2023-winners/}{\faLink[regular]}
}

\vspace{0.1cm}

\experience{Best Development Award}{PoliMi Capstone Projects 2022}{}{Apr 2022}{
SPINTEST - Data-Driven Compliancy Assessment for Extra-Corporeal Centrifual Blood Pumps \href{https://github.com/alberto-rota/ECC-Centifugal-Pump-Tester}{\faGithub}
}

% ----- Selected Papers -----
\section{Selected Research Papers}
\vspace{-0.55cm}
\noindent\begin{tabular}{@{} >{\raggedleft}p{0.055\linewidth}
 p{0.875\linewidth}}
& \bibliographystyle{unsrt} % We choose the "plain" reference style
\setlength{\labelsep}{0.5cm}
\bibliography{papers.bib} \\
\end{tabular}

\nocite{fu2023recent}
\section{Disclosures}

\noindent\begin{tabular}{@{} >{\raggedleft}p{0.1\linewidth}
 p{0.875\linewidth}}
 
\textit{GDPR} & I authorize the processing of personal data according to EU Regulation 679/2016 or according to the reader's local regulations if not in the EU \\
\textit{Accessibility} & I authorize the publication and the complete accessibility of this CV according to the italian D. Lgs n. 33 of March 14 2013 \\
\textit{NDA} & Research work in this CV tagged with [NDA] has been carried out under IP protection policies and a Non-Disclosure Agreement. Details available upon request and on a subject basis. \\
\end{tabular}

\end{document}